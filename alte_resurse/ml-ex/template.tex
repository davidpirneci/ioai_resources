\documentclass{article}
\usepackage{amsmath,amsfonts,amssymb}

\begin{document}

\section*{Problema 1}

\section*{Solutie}

a) Fie $d = \text{nr. de zone patratice de dimensiune } 16 \times 16 \text{ posibile}$ $(d = 256^{16 \times 16})$.\\
Fie $\phi: X \rightarrow \mathbb{R}^d$, $\phi(x) = [\phi_{p_1}(x), \phi_{p_2}(x), \phi_{p_3}(x), \dots, \phi_{p_d}(x)]$, unde \\\\ $\phi_{p_k}(x) =
\begin{cases}
1 & \text{dacă } x \text{ conține zona patratica } p_k \\
0 & \text{în caz contrar}
\end{cases}$\\\\\\

\noindent $\phi(x) \cdot \phi(x') = \sum_{i=1}^d \phi_{p_i}(x) \cdot \phi_{p_i}(x')$, care reprezintă numărul de zone patratice comune imaginii $x$ și $x' = \phi(x) \cdot \phi(x') = k_1(x, x')$ \\\\

\noindent b) Fie $n = 3$ și $k_2(x_1, x_2) = 1, k_2(x_2, x_3) = 0 , k_2(x_1, x_3) = 1$ \\

\noindent $G =
\begin{bmatrix}
k_2(x_1, x_1) & k_2(x_1, x_2) & k_2(x_1, x_3) \\
k_2(x_2, x_1) & k_2(x_2, x_2) & k_2(x_2, x_3) \\
k_2(x_3, x_1) & k_2(x_3, x_2) & k_2(x_3, x_3)
\end{bmatrix} =
\begin{bmatrix}
1 & 1 & 1 \\
1 & 1 & 0 \\
1 & 0 & 1
\end{bmatrix}$\\\\

\noindent Fie $f =
\begin{bmatrix}
x \\ y \\ z
\end{bmatrix}$, $x, y, z \in \mathbb{R}$\\
$f^T G f =
\begin{bmatrix}
x & y & z
\end{bmatrix}
\begin{bmatrix}
1 & 1 & 1 \\
1 & 1 & 0 \\
1 & 0 & 1
\end{bmatrix}
\begin{bmatrix}
x \\ y \\ z
\end{bmatrix} =
\begin{bmatrix}
x & y & z
\end{bmatrix}
\begin{bmatrix}
x + y + z \\ x + y \\ x + z
\end{bmatrix} =
x^2 + x y + x z + x y + y^2 + z^2 + z x$\\

\noindent Fie $x = -1, y = 1, z = 1 \Rightarrow f^T G f = 1 - 1 - 1 - 1 + 1 + 1 - 1  = -1 \Rightarrow \exists f \in \mathbb{R}^3,$ astfel incat $f^T G f < 0$ (generalizarea se realizează prin inducție).

\end{document}
